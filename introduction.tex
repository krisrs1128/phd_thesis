\chapter{Introduction}
\label{ch:introduction}

A single question lies behind the research efforts of both the data
visualization and statistical modeling communities: What are the most effective
techniques for identifying and representing latent structure in data? The
problem is that even moderately large collections of data are difficult to
mentally process -- some reduction, some more succinct representation, is
necessary before the data can be used to guide reasoning. In visualization, a
representation is a group of markings on a page (or screen), while in
statistics, a representation is a set of mathematical objects, fitted parameters
or functions, for example. In spite of the differences in the substrates from
which these representations are molded, the visualization and statistics
communities have arrived at many similar principles for guiding this reduction.
For example, an information-dense figure describes a large collection of numbers
in a small physical space on the page \citep{tufte2014visual}, while a
sufficient statistic encapsulates everything about a data set necessary to
describe a data generating mechanism. Similarly, hierarchical modeling, in which
many parameters share information via a hyperprior, has a natural counterpart in
the small-multiples principle, in which many versions of a figure are created,
for small units drawn from some population \citep{gelman2004exploratory}. This
thesis blends ideas from both the modeling and visualization communities to make
the representation of latent structure more accessible and automatic for the
practicing data analyst.

\begin{figure}
  \centering
  \includegraphics{figure/intro_mean}
  \includegraphics{figure/intro_smooth}
  \caption{Summaries based on user interaction and function fits based on some
    smoothness criterion are natural data reductions when working from
    visualization and statistical perspectives, respectively.}
\end{figure}

Data is never analyzed in a vacuum. Its collection and study is only valuable
as far as it helps resolve important ambiguities in the systems of interest. To
ground our study, we focus on applications to microbiome data, seeking
representations that we believe will simplify the investigation of a variety of
microbiome-related questions.

The essential contribution of this thesis is to streamline and democratize the
discovery and visualization of latent structure in the microbiome. Concretely,
this involves several lines of study,
\begin{itemize}
  \item Designing example workflows: There are many possible approaches for a
    microbiome analysis pipeline, from raw data to model criticism, but few
    references for how to choose between options and assemble a coherent
    workflow. Chapter \ref{ch:text_analysis} offers some basic guideposts.
  \item Developing software packages: Sometimes the same conceptually complex or
    time-consuming representation task appears repeatedly across studies. This
    has motivated the creation of packages to simplify these difficult steps,
    several of which are reviewed in Chapter \ref{ch:interactive_vis}.
  \item Distilling relevant literature: Sometimes the barrier to effective
    analysis is not the implementation of a technique, but knowledge of which
    methods are relevant and effective. This is especially the case for more
    complex analysis questions, and is the underlying motivation for
    Chapters \ref{ch:multitable} and \ref{ch:dynamic_regimes}.
\end{itemize}
Our hope is that this work empowers the microbiome community to make full use of
ideas developed in statistics and data visualization.

In more detail, Chapter \ref{ch:text_analysis} compares several probabilistic
models for representing longitudinal data, relying on simulations and a case
study. In particular, an analogy between microbiome and document modeling is
developed, and structured models for text are applied and assessed on simulated
and real microbiome data. Further, steps for visual characterization of fitted
parameters and formal quantitative diagnostics are illustrated.

Chapter \ref{ch:interactive_vis} introduces three new data visualization
packages, \texttt{treelapse}, \texttt{centroidview}, and \texttt{mvarVis}. Each
is designed with a particular visual comparison in mind -- for example,
\texttt{treelapse} and \texttt{centroidview} address the tree-structured
differential abundance and dynamics problems, while \texttt{mvarVis} simplifies
comparisons between ordinations. These packages are designed with models for
latent structure in mind, and they can guide both model design and evaluation.
Each package has an interactive component, and we find this interactivity
useful for drawing attention to different views of the data with minimal effort.

Chapter \ref{ch:multitable} is focused on the problem of integrating multiple
data sources into a unified representation of between-table covariation. This
situation is frequently encountered in modern microbiome studies, where
participants provide samples\footnote{Throughout this thesis, we use the term
  ``sample'' to refer to a physical measurement, reflecting the usage of the
  term in biological applications. They need not be samples from some
  distribution, in the statistical sense.} to be processed by several 'omics
technologies at once. We intentionally focus our efforts on surveying and
providing code / visualizations of existing methodology, rather than embarking
on new algorithm design. Our basic premise is that the most substantial barrier
to effective data analysis in applied fields is not the rate at which good
algorithms can be designed and implemented. Rather, the essential difficulty
stems from a failure to clearly articulate and compare existing methods. In a
way, this work work is the computational descendant of classical optimality and
efficiency comparisons: Given a morass of potential analyses in a practical
problem, pave a clear and well-justified path forwards.

In a similar spirit, chapter \ref{ch:dynamic_regimes} reviews literature on
discovering latent dynamic regimes in time series. This type of reduction is
relevant in studies of stability and change in the microbiome, a topic of
ecological and medical interest. We begin our survey with heuristic, but
relatively inflexible approaches, and then alternately consider techniques for
explicitly accounting for different types of structure -- counts, shared
changepoints, and zero-inflation, for example. Again, we place emphasis on
visualization, evaluation, and the statistical and computational trade-offs
implicit in different modeling approaches.

A key component of the scholarship in this thesis takes the form of computer
code, all of which is publicly available, with links appearing within each
chapter.
