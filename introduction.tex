\chapter{Introduction}
\label{ch:introduction}

A single question lies behind the research efforts of both the data
visualization and statistical modeling communities: What are the most effective
techniques for identifying and representing latent structure in data? The
problem is that even moderately large collections of data are difficult to
mentally process -- some reduction, some more succinct representation, is
necessary before the data can be used to guide reasoning. In visualization, a
representation is a group of markings on a page (or screen), while in
statistics, a representation is set of mathematical objects, fitted parameters
or functions, for example. In spite of the differences in the substrates from
which these representations are molded, the visualization and statistics
communities have arrived at many similar principles for guiding this reduction.
For example, an information-dense figure describes a large collection of numbers
in a small physical space on the page \citep{tufte2014visual}, while a
sufficient statistic encapsulates everything about a data set necessary to
estimate a parameter. This thesis blends ideas from both communities to make the
representation of latent structure more accessible and automatic for the
practicing data analyst.

\begin{figure}
  \centering
  \includegraphics{figure/intro_mean}
  \includegraphics{figure/intro_smooth}
  \caption{Summaries based on user interaction and function fits based on some
    smoothness criterion are natural data reductions when working from
    visualization and statistical perspectively, respectively.}
\end{figure}

Data is never analyzed in a vacuum. Its collection and study is only valuable
as far as it helps resolve important ambiguities in the systems of interest. To
ground our study, we focus on applications to microbiome data, seeking
representations that we believe will simplify the investigation of a variety of
microbiome-related questions.

The essential contribution of this thesis is to streamline and democratize the
discovery and visualization of latent structure in the microbiome. Concretely,
this involves several lines of study,
\begin{itemize}
  \item Designing example workflows: There are many approaches possible to a a
    microbiome analysis pipeline, from raw data to model criticism, but few
    references for how to choose between options and assemble a coherent
    workflow. Chapter \ref{ch:text_analysis} offers some basic guideposts.
  \item Developing software packages: Sometimes the same conceptually complex or
    time-consuming representation task appears repeatedly across studies. This
    has motivated the creation of packages to simplify these difficult steps,
    several of which are reviewed in Chapter \ref{ch:interactive_vis}.
  \item Distilling relevant literature: Sometimes the barrier to effective
    analysis is not the implementation of a technique, but knowledge of which
    methods are relevant and effective. This is especially the case for more
    specialized analysis questions, and is the underlying motivation for
    Chapters \ref{ch:multitable} and \ref{ch:dynamic_regimes}.
\end{itemize}
Our hope is that this work empowers the microbiome community to make full use of
ideas developed in statistics and data visualization.

In more detail, Chapter \ref{ch:text_analysis} compares several probabilistic
models for representing longitudinal data, relying on simulations and a case
study. Chapter \ref{ch:interactive_vis} introduces three new data visualization
packages, \texttt{treelapse}, \texttt{centroidview}, and \texttt{mvarvis}. Each
is designed with a particular visual comparison in mind -- for example,
\texttt{treelapse} and \texttt{centroidview} address the tree-structured
differential abundance and dynamics problems, while \texttt{mvarVis} simplifies
comparisons between ordinations. Chapter \ref{ch:multitable} is focused on the
problem of integrating multiple data sources into a unified representation of
between-table covariation. This situation is frequently encountered in modern
microbiome studies, where participants provide samples to be processed by
several 'omics technologies at once. Finally, chapter \ref{ch:dynamic_regimes}
reviews literature on discovering latent dynamic regimes in time series. This
type of reduction is relevant in studies of stability and change in the
microbiome, a topic of ecological and medical interest.

A key component of the scholarship in this thesis takes the form of computer
code, all of which is publicly available, with links appearing within each
chapter.
