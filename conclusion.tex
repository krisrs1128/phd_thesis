\chapter{Conclusion}
\label{ch:conclusion}

In this thesis, we have evaluated workflows, developed software, and distilled
literature relevant to discovery and visualization of latent structure in the
microbiome.

In Chapter \ref{ch:interactive_vis}, we developed a collection of packages that
made use of data visualization concepts to ease pain-points in microbiome data
analysis. The package \texttt{treelapse} addresses the problem of comparing time
series that can be arranged hierarchically, borrows the focus-plus-context and
linked brushing ideas, common in the data visualization literature, to make it
easy to ask which parts of a phylogenetic tree contain trajectories of a given
shape. Conversely, we facilitate querying subsets of a tree to see what
trajectories match those nodes. This interactivity greatly speeds up the process
of exploring species dynamics, which can otherwise involve printing hundreds of
species trajectory plots.

The package \texttt{centroidview} offers a different take on the subtree
visualization problem, more directly suited to the problem of comparing
centroids corresponding to subtrees in a hierarchical clustering. It provides a
few additional visual elements -- color, a heatmap, and comparison barplots --
that enrich a \texttt{treelapse}-type display for characterizing clusters at
different levels of granularity.

Chapter \ref{ch:text_analysis} evaluates workflows for probabilistically
modeling microbiome data inspired by literature in the text analysis literature.
We describe probabilistic representations of the sample-by-species abundance
matrix, which we view as analogous to a document-term matrix. While the analogy
between document-term and species abundance matrices had been observed before,
the only previous papers applying text analysis models to the microbiome failed
to visualize model output (focusing on test performance instead) or provide code
and data. In addition to making this workflow easily accessible, we perform
simulations and model criticism for several different approaches, which offers
guidance on the appropriateness of alterantive models to microbiome analysis.

Chapter \ref{ch:multitable} reviews the problem of relating several tables to
one another, motivated by an application to a study comparing body fat
composition to bacterial species abundances, with interest in connections
between the microbiome and host metabolism. If there were only one summary of
body fat -- for example, BMI -- this is simply a high-dimensional regression
problem. However, this study provides fat and muscle mass measurements at
various locations across the body, in order to provide a richer portrait of body
fat composition than simply BMI. There are many proposals for studying such
multitable data, developed across various literatures, including statistics,
ecology, and machine learning, for example. This chapter provides a unified
review, along with applications to the body composition data, with the overall
goals of comparison and distillation.

In Chapter \ref{ch:dynamic_regimes}, we formulate the regime detection problem
in the context of microbiome studies and review associated methods. The problem
here is to identify time segments where species trajectories switch behavior. It
is also of interest here to characterize subsets of species with similar changes
in dynamics. In our survey, we describe statistical and machine learning
techniques, including those based on regression trees, Bayesian nonparametric
Hidden Markov Models, Gaussian Processes, and changepoint detection. We discuss
differences in underlying assumptions, modeling performance, computational
efficiency, and types of output made available.

This thesis lays the groundwork for potential projects related to visualization
and workflow evaluation of biological data. Towards a richer suite of
visualization tools, the focus-plus-context and linking principles could be
applied broadly to a variety of data structures and scientific questions. For
example, the study of networks can shed light on basic problems of microbial
ecology and genetics \citep{akutsu2000inferring, coyte2015ecology}, and
networks, like trees, are difficult to study at multiple scales. It is easy to
imagine variants of DOI-trees that collapse tightly-knit communities in a
network. This could provide an interactive analog of a hierarchical clustering
of a network adjacency matrix, a type of visual multiscale analysis. Further,
these networks often evolve over time \citep{kolar2010estimating}, and the
linking ideas advocated in the \texttt{treelapse} package offer a path towards
incorporating additional temporal structure.

Similarly, the linking idea, is naturally applicable to the multitable problems
reviewed in Chapter \ref{ch:multitable}. We could imagine linked versions of
sparse CCA, for example, which describe the variation in one table conditional
on the sparse CCA scores of another. Indeed, there could be much to be gained by
combining the automatic reductions provided by statistical methods with the
ease-of-navigation offered by interactive visualization.

The workflow and distillation ideas in Chapters \ref{ch:text_analysis},
\ref{ch:multitable}, and \ref{ch:dynamic_regimes} could be developed in several
directions. In Chapter \ref{ch:text_analysis}, we relied upon posterior
predictive checks to evaluate the quality of Bayesian model fits. Recently,
several more quantitative measures of model quality \citep{gorham2015measuring,
  liu2016kernelized} and robustness \citep{giordano2016fast} have become
available, and applying them to model criticism in the workflows of this thesis
would be worthwhile. Further, wrapping these methods in fully documented R
packages, would facilitate more straightforwards experimentation by data
analysis practitioners. Finally, blending ideas between methods across chapters
could lead to useful hybrid algorithms. For example, it is sometimes necessary
to perform regime detection in multitable data, or to consider DTM-style
zero-inflation across clustered time series.

We expect that the foundation laid out by this thesis will be relevant to the
creation and evaluation of methods related to data visualization and latent
variable modeling, as these fields continue to develop.
