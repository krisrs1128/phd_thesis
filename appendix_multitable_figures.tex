\chapter{Supplementary Figures for Multitable Methods}
\label{ch:multitable_supp_figs}

This appendix includes figures complementing those in Chapter
\ref{ch:multitable}.

\begin{figure}
  \centering
  \includegraphics[width=0.3\textwidth]{figure/multitable/pca/proj_plot_2}
  \caption{This is the analog of Figure \ref{fig:pca-approx} in the case that
    PCA is run on all body composition variables, rather than just lean and
    fat mass. That is, we project the original values for these two features
    onto the top two PCs obtained from a PCA on all body composition variables.
    Since the PCA is working in a large space, the projected points are
    generally not too far from their original positions. However, note that one
    outlier on the far right is projected into the bulk of points in the center
    -- the variation coming from this one point is too specific to be preserved
    by PCA.
  \label{fig:pca-approx-2}}
\end{figure}

\begin{figure}
  \centering
  \includegraphics[width=0.3\textwidth]{figure/multitable/pca/proj_plot_3}
  \caption{Here we perform the same procedure as in Figure
    \ref{fig:pca-approx-2}, except instead of projecting onto the top two PCs,
    we project onto only the top PC. The main point is that, while in two
    dimensions (Figure \ref{fig:pca-approx}), the behavior of the projection is
    easy to understand in terms of orthogonal errors, the corresponding
    orthogonal projection in higher dimensions is more complex.
  \label{fig:pca-approx-3}}
\end{figure}

\begin{figure}
  \centering \includegraphics[width=\textwidth]{figure/multitable/coia/loadings}
  \caption{These are the loadings obtained from CoIA, which are analogous to
    those obtained from concatenated PCA (Figure \ref{fig:loadings}) and CCA
    (Figure \ref{fig:cca_loadings}). \label{fig:coia_loadings} }
\end{figure}

\begin{figure}
  \centering
  \includegraphics[width=\textwidth]{figure/multitable/coia/scores_rl_ratio}
  \caption{These the same CoIA scores as in Figure \ref{fig:coia_scores_android_fm},
    but shaded instead by Ruminococcaceae / Lachnospiraceae ratio, as in Figures
    \ref{fig:scores_rl_ratio} and
    \ref{fig:cca_scores_rl_ratio}. \label{fig:coia_scores_rl_ratio} }
\end{figure}

\begin{figure}
  \centering
  \includegraphics[width=\textwidth]{figure/multitable/pca_iv/scores_android_fm}
  \caption{We can display the scores obtained by PCA-IV. The results are similar
    to those from combined-PCA, which is not surprising, since they are related
    to the PCA based only on microbial
    abundances. \label{fig:pca_iv_scores_android_fm} }
\end{figure}

\begin{figure}
  \centering
  \includegraphics[width=\textwidth]{figure/multitable/pca_iv/scores_rl_ratio}
  \caption{This provides the same scores as Figure
    \ref{fig:pca_iv_scores_android_fm}, but shaded by Ruminoccoccaceae vs.
    Lachnospiraceae ratio. \label{fig:pca_iv_scores_rl_ratio} }
\end{figure}

\begin{figure}
  \centering
  \includegraphics[width=\textwidth]{figure/multitable/pmd/illustration_scatter}
  \caption{An alternative view of Figure \ref{fig:pmd_illustration_sequence},
    where the two underlying sources are plotted in two dimensions rather than
    one. Basic unidentifiabilities in the model are visible as long swaps
    between true and recovered points. \label{fig:pmd_illustration_scatter} }
\end{figure}

 \begin{figure}
   \centering
   \includegraphics[width=0.9\textwidth]{figure/multitable/lda_cca/unshared_scores_fm_posterior}
   \caption{Posterior samples of scores $\xi^{Y}_i$ associated with the body
     composition variables. These are much more spread out than either the
     $\xi_{i}^s$ or $\xi_{i}^Y$, likely because each scores is based on just the
     36 body composition variables, rather than counts for all $\sim 400$
     species.
     \label{fig:lda_cca_unshared_scores_fm_posterior} }
 \end{figure}

 \begin{figure}
   \centering
   \includegraphics[width=0.9\textwidth]{figure/multitable/lda_cca/shared_scores_fm_posterior}
   \caption{Posterior samples of scores $\xi^{s}_i$ shared across both bacterial
     abundance and body composition variables. Points are positioned according
     to their value in the first two dimensions shared body composition loading
     dimensions, $B^{Y}_{\cdot, 1:2}$. Clouds of points correspond to
     participants, and they are shaded by their android fat mass. This method
     does not appear to pick up on any associations between body composition and
     microbiome structure.
 \label{fig:lda_cca_shared_scores_fm_posterior} }
 \end{figure}

\begin{figure}
  \centering
  \includegraphics[width=\textwidth]{figure/multitable/graph_lasso/multitask_lasso_hm_lambdas}
  \caption{ Inspecting coefficients from multitask lasso across $\lambda$
    regularization parameters highlights species with different directions and
    strengths of association with body composition variables. Different panel
    columns indicate different taxonomic families, while rows correspond to
    different response variables. Individual tiles give the coefficients of an
    individual species (column) against a response at a particular $\lambda$
    (row). Within panels, the $\lambda$s are sorted from least to most
    regularization. Species have been clustered according to their lasso
    coefficients, revealing groups of species with similar coefficient profiles
    across groups of responses.
    \label{fig:graph_lasso_multitask_lasso_hm_lambdas} }
\end{figure}
