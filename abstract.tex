\prefacesection{Abstract}

Human microbiomes -- the collections of bacteria living around and within the
human body -- are complex ecological systems, and describing their structure and
function in different contexts is important from both basic scientific and
medical perspectives. Viewed through a statistical lens, many microbiome
analysis goals can framed in terms of discovering and describing latent
structure. For example, this structure might reflect sudden environmental shocks
that affect certain subsets of species, or may illuminate gradual shifts in
community composition. In this thesis, we survey and develop ideas from the data
visualization and probabilistic modeling literatures that we have found useful
in identifying and characterizing such structure in the microbiome.

On the data visualization front, we describe the focus-plus-context and linking
principles, and describe new R packages that use these ideas to facilitate
visualization of hierarchical collections of time series. These tools streamline
the navigation of complex data, guiding researchers towards plausible
statistical models.

We then turn our attention to modeling, motivated by the fact that microbiome
species abundance data often have effectively low-dimensional evolutionary,
temporal, and count structure. We then characterize and review methods
appropriate for three classes of common microbiome data analysis problems --
dimensionality reduction, multitable integration, and regime detection. For
dimensionality reduction, we explore basic probabilistic latent variable models,
focusing on mixed-membership and matrix factorization techniques. For multitable
integration, we contrast nonparametric ordination, structured regularization,
and probabilistic modeling approaches. For regime detection, we compare variants
of hidden markov, dynamical systems, and changepoint models, along with
baselines that don't take into account time structure.

Throughout, we illustrate visualization and modeling techniques using real human
gut microbiome data. Code and data for all experiments is available publicly
online.
